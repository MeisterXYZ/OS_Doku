%%%%%%%%%%%%%%%%%%%%%%%%%%%%%%%%%%%%%%%%%
% Journal Article
% LaTeX Template
% Version 1.4 (15/5/16)
%
% This template has been downloaded from:
% http://www.LaTeXTemplates.com
%
% Original author:
% Frits Wenneker (http://www.howtotex.com) with extensive modifications by
% Vel (vel@LaTeXTemplates.com)
%
% License:
% CC BY-NC-SA 3.0 (http://creativecommons.org/licenses/by-nc-sa/3.0/)
%
%%%%%%%%%%%%%%%%%%%%%%%%%%%%%%%%%%%%%%%%%

%----------------------------------------------------------------------------------------
%	PACKAGES AND OTHER DOCUMENT CONFIGURATIONS
%----------------------------------------------------------------------------------------

\documentclass[twoside,twocolumn]{article}

\usepackage{blindtext} % Package to generate dummy text throughout this template 

\usepackage[sc]{mathpazo} % Use the Palatino font
\usepackage[T1]{fontenc} % Use 8-bit encoding that has 256 glyphs
\linespread{1.05} % Line spacing - Palatino needs more space between lines
\usepackage{microtype} % Slightly tweak font spacing for aesthetics

\usepackage[german]{babel} % Language hyphenation and typographical rules

\usepackage[hmarginratio=1:1,top=32mm,columnsep=20pt]{geometry} % Document margins
\usepackage[hang, small,labelfont=bf,up,textfont=it,up]{caption} % Custom captions under/above floats in tables or figures
\usepackage{booktabs} % Horizontal rules in tables

\usepackage{lettrine} % The lettrine is the first enlarged letter at the beginning of the text

\usepackage{enumitem} % Customized lists
\setlist[itemize]{noitemsep} % Make itemize lists more compact

\usepackage{abstract} % Allows abstract customization
\renewcommand{\abstractnamefont}{\normalfont\bfseries} % Set the "Abstract" text to bold
\renewcommand{\abstracttextfont}{\normalfont\small\itshape} % Set the abstract itself to small italic text

\usepackage{titlesec} % Allows customization of titles
\renewcommand\thesection{\Roman{section}} % Roman numerals for the sections
\renewcommand\thesubsection{\roman{subsection}} % roman numerals for subsections
\titleformat{\section}[block]{\large\scshape\centering}{\thesection.}{1em}{} % Change the look of the section titles
\titleformat{\subsection}[block]{\large}{\thesubsection.}{1em}{} % Change the look of the section titles

\usepackage{fancyhdr} % Headers and footers
\pagestyle{fancy} % All pages have headers and footers
\fancyhead{} % Blank out the default header
\fancyfoot{} % Blank out the default footer
\fancyhead[C]{Running title $\bullet$ May 2016 $\bullet$ Vol. XXI, No. 1} % Custom header text
\fancyfoot[RO,LE]{\thepage} % Custom footer text

\usepackage{titling} % Customizing the title section

\usepackage{hyperref} % For hyperlinks in the PDF


\usepackage[utf8]{inputenc}

%----------------------------------------------------------------------------------------
%	TITLE SECTION
%----------------------------------------------------------------------------------------

\setlength{\droptitle}{-4\baselineskip} % Move the title up

\pretitle{\begin{center}\Huge\bfseries} % Article title formatting
\posttitle{\end{center}} % Article title closing formatting
\title{Data Lakes} % Article title
\author{%
\textsc{Raphael Drechsler}\\[1ex] % Your name
\normalsize HTWK Leipzig \\ 
\normalsize Fakultät Informatik, Mathematik und Naturwissenschaften\\ 
\normalsize Studiengang Informatik Master - Matrikelnr. 69872\\% Your email address
%\and % Uncomment if 2 authors are required, duplicate these 4 lines if more
%\textsc{Jane Smith}\thanks{Corresponding author} \\[1ex] % Second author's name
%\normalsize University of Utah \\ % Second author's institution
%\normalsize \href{mailto:jane@smith.com}{jane@smith.com} % Second author's email address
}
 \date{30.30.2018} % Leave empty to omit a date
\renewcommand{\maketitlehookd}{%
\begin{abstract}
\noindent 
Begriff ist 2010 entstanden und in den letzten Jahren gehyped (Quellen 1,2,3)
Anschließend gab es Kritik bis hin zu einem Artikel über Fake News? (Quelle 4)
Zum Zeitpunkt der Rechereche kein deutsch-Sprachiger Artikel auf Wiki
Motivation besteht darin diese Unklarheiten zu beleuchten, klären was ein Data-Lake ist und sich mit der Frage ''Fake-News'' auseinanderzusetzen.


\end{abstract}
}

%----------------------------------------------------------------------------------------

\begin{document}

% Print the title
\maketitle

%----------------------------------------------------------------------------------------
%	ARTICLE CONTENTS
%----------------------------------------------------------------------------------------

\section{Definitionsfrage ''Data Lake''}
\lettrine[nindent=0em,lines=2]{J} etzt gehts los.
\textbf{DRT:}
Kein Akademischer Ursprung. Dixon hats gemacht.
\cite{src1}
\cite{src2}
\cite{src3}
\cite{src4}
\cite{src5}
\cite{src6}
\cite{src7}
\cite{src8}
\cite{src9}
\cite{src10}
\cite{src11}
\cite{src12}
\cite{src13}
\cite{src14}
\cite{src15}

\subsection{Definition nach Dixon}
Quellen: 5,6
\subsection{Pentaho-Architektur 2010}
Quellen: 6
\subsection{Begriffs-Chaos}
Quellen: 7

\section{Wie funktioniert ein Data Lake?}
\subsection{Aufbau}
QUELLEN : 8, 9
\subsection{Workflow}
Allgemein und die Rolle des Data Scientist
QUELLEN : 8, 9
\paragraph{Storage}
QUELLEN : 8, ?
\paragraph{Ingestion}
QUELLEN : 8, 10, 11
\paragraph{Process}
QUELLEN : 8, 12
\paragraph{Consumption}
QUELLEN : 8, 12
\paragraph{Monitoring}
QUELLEN : 8, ?
\paragraph{Data Governance}
QUELLEN : 8, 12


\section{Data Swamps: Kritik am Data Lake}
Mögliche Darstellungen die es so gibt:
\begin{itemize}
	\item Sumpf
	\item Finnland
	\item Flohmarkt
\end{itemize}

Gartners wesentliche Punkte\\

Battle: Gartner vs. Dixon und \\
QUELLEN : 3, 12, 13\\
DL revisited $\rightarrow$ ungenaue Definition ein Problem zu kritisieren oder nur ein Kritikpunkt?\\
Quellen: 14,15,

In Praxis dann Sean Martin zitieren.\\
Trend des Vorsichtiger werdens und die Flut kommen sehen. Paradigmenwechsel.
Quelle: 1

\section{Fake-News! Existieren Data Lakes überhaupt?}
Blogeintrag nur nennen und erste Zeile zitieren.\\
Quelle: 4

Nach Recherche lassen sich da schon ein paar Firmen finden, die Data Lake Lösungen anbieten.
$\rightarrow$auflisten


Nach weiterer Recherche auch success-stories auffindbar. Also irgendwie schon.


Die wesentliche Frage ist allerdings die Definitionsfrage. Selber Schluss im Blogeintrag. Lösung die dem Paradigma grundlegend folgen gibt es. Jetzt im Auge des Betrachters ob man das Kind beim Namen nennt oder nicht.




%------------------------------------------------

\section{Methods}

Maecenas sed ultricies felis. Sed imperdiet dictum arcu a egestas. 
\begin{itemize}
\item Donec dolor arcu, rutrum id molestie in, viverra sed diam
\item Curabitur feugiat
\item turpis sed auctor facilisis
\item arcu eros accumsan lorem, at posuere mi diam sit amet tortor
\item Fusce fermentum, mi sit amet euismod rutrum
\item sem lorem molestie diam, iaculis aliquet sapien tortor non nisi
\item Pellentesque bibendum pretium aliquet
\end{itemize}
\blindtext % Dummy text

Text requiring further explanation\footnote{Example footnote}.

%------------------------------------------------

\section{Results}

\begin{table}
\caption{Example table}
\centering
\begin{tabular}{llr}
\toprule
\multicolumn{2}{c}{Name} \\
\cmidrule(r){1-2}
First name & Last Name & Grade \\
\midrule
John & Doe & $7.5$ \\
Richard & Miles & $2$ \\
\bottomrule
\end{tabular}
\end{table}

\blindtext % Dummy text

\begin{equation}
\label{eq:emc}
e = mc^2
\end{equation}

\blindtext % Dummy text

%------------------------------------------------



%----------------------------------------------------------------------------------------
%	REFERENCE LIST
%----------------------------------------------------------------------------------------

\bibliographystyle{unsrt}
\bibliography{bib.bib} % The file containing the bibliography

%----------------------------------------------------------------------------------------

\end{document}
