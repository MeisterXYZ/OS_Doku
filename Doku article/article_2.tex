%%%%%%%%%%%%%%%%%%%%%%%%%%%%%%%%%%%%%%%%%
% Journal Article
% LaTeX Template
% Version 1.4 (15/5/16)
%
% This template has been downloaded from:
% http://www.LaTeXTemplates.com
%
% Original author:
% Frits Wenneker (http://www.howtotex.com) with extensive modifications by
% Vel (vel@LaTeXTemplates.com)
%
% License:
% CC BY-NC-SA 3.0 (http://creativecommons.org/licenses/by-nc-sa/3.0/)
%
%%%%%%%%%%%%%%%%%%%%%%%%%%%%%%%%%%%%%%%%%

%----------------------------------------------------------------------------------------
%	PACKAGES AND OTHER DOCUMENT CONFIGURATIONS
%----------------------------------------------------------------------------------------

\documentclass[twoside,twocolumn]{article}

\usepackage{blindtext} % Package to generate dummy text throughout this template 

\usepackage[sc]{mathpazo} % Use the Palatino font
\usepackage[T1]{fontenc} % Use 8-bit encoding that has 256 glyphs
\linespread{1.05} % Line spacing - Palatino needs more space between lines
\usepackage{microtype} % Slightly tweak font spacing for aesthetics

\usepackage[german]{babel} % Language hyphenation and typographical rules

\usepackage[hmarginratio=1:1,top=32mm,columnsep=20pt]{geometry} % Document margins
\usepackage[hang, small,labelfont=bf,up,textfont=it,up]{caption} % Custom captions under/above floats in tables or figures
\usepackage{booktabs} % Horizontal rules in tables

\usepackage{lettrine} % The lettrine is the first enlarged letter at the beginning of the text

\usepackage{enumitem} % Customized lists
\setlist[itemize]{noitemsep} % Make itemize lists more compact

\usepackage{abstract} % Allows abstract customization
\renewcommand{\abstractnamefont}{\normalfont\bfseries} % Set the "Abstract" text to bold
\renewcommand{\abstracttextfont}{\normalfont\small\itshape} % Set the abstract itself to small italic text

\usepackage{titlesec} % Allows customization of titles
\renewcommand\thesection{\Roman{section}} % Roman numerals for the sections
\renewcommand\thesubsection{\roman{subsection}} % roman numerals for subsections
\titleformat{\section}[block]{\large\scshape\centering}{\thesection.}{1em}{} % Change the look of the section titles
\titleformat{\subsection}[block]{\large}{\thesubsection.}{1em}{} % Change the look of the section titles

\usepackage{fancyhdr} % Headers and footers
\pagestyle{fancy} % All pages have headers and footers
\fancyhead{} % Blank out the default header
\fancyfoot{} % Blank out the default footer
\fancyhead[C]{Running title $\bullet$ May 2016 $\bullet$ Vol. XXI, No. 1} % Custom header text
\fancyfoot[RO,LE]{\thepage} % Custom footer text

\usepackage{titling} % Customizing the title section

\usepackage{hyperref} % For hyperlinks in the PDF


\usepackage[utf8]{inputenc}
\usepackage{graphicx}

%----------------------------------------------------------------------------------------
%	TITLE SECTION
%----------------------------------------------------------------------------------------

\setlength{\droptitle}{-4\baselineskip} % Move the title up

\pretitle{\begin{center}\Huge\bfseries} % Article title formatting
\posttitle{\end{center}} % Article title closing formatting
\title{Data Lakes} % Article title
\author{%
\textsc{Raphael Drechsler}\\[1ex] % Your name
\normalsize HTWK Leipzig \\ 
\normalsize Fakultät Informatik, Mathematik und Naturwissenschaften\\ 
\normalsize Studiengang Informatik Master - Matrikelnr. 69872\\% Your email address
%\and % Uncomment if 2 authors are required, duplicate these 4 lines if more
%\textsc{Jane Smith}\thanks{Corresponding author} \\[1ex] % Second author's name
%\normalsize University of Utah \\ % Second author's institution
%\normalsize \href{mailto:jane@smith.com}{jane@smith.com} % Second author's email address
}
 \date{30.30.2018} % Leave empty to omit a date
\renewcommand{\maketitlehookd}{%
\begin{abstract}
\noindent 
Der Begriff des Data Lakes ist 2010 entstanden und wurde in den letzten Jahren stark ''gehyped''.\cite{src1} \cite{src2} \cite{src3} Es haben sich viele verschiedene Konzepte und Ansichten zum Thema entwickelt. Im Internet findet man bei einer Recherche zum Thema Data Lake von einem existierneden Unternehmen, welches sich ''the Data-Lake-Company'' nennt\cite{dlc}, bis hin zu einem Blogeintrag, der die Frage ''Are Data Lakes Fake-News?'' mit ja beantwortet\cite{src4} eine ganze Menge. Dabei wird die Frage danach, was ein Data Lake ist, von den verscheidenen Quellen nicht eindutig beantwortet. Auch gibt es zum Zeitpunkt des Erstellens dieses Dokumentes in der deutschsprachigen Wikipedia nock keinen Eintrag zu diesem Thema. Die Motivation dieses Abstracts besteht also darin, die bestehenden Unklarheiten zu beleuchten; zu klären was ein Data-Lake ist und sich mit der Frage ''Are Data Lakes Fake-News'' auseinanderzusetzen.
\end{abstract}
}

%----------------------------------------------------------------------------------------

\begin{document}

% Print the title
\maketitle

%----------------------------------------------------------------------------------------
%	ARTICLE CONTENTS
%----------------------------------------------------------------------------------------

\section{Definitionsfrage ''Data Lake''}
\lettrine[nindent=0em,lines=2]{J} etzt gehts los.
\textbf{DRT:}
Kein Akademischer Ursprung. Dixon hats gemacht. Quellen sind dabei sein Blog\cite{src5} und YT\cite{src6}. Bei YT wird das Produkt von Pentaho vorgestellt. In diesem Rahmen wird der Begriff geboren.
Dixon's Betrachtung beginnt datmit, dass Dixon Big-Data Szenarios betrachtet. Feststellen von Eigenschaften. Diese sind:
\begin{itemize}
	\item bla
	\item lo
	\item li
\end{itemize}

Dazu noch im Wesentlichen die Eigenschafen, dass unbekannte Fragen beantw können und keine 1 mio und verscheidene Anwender Bisschen das kleinfieh Prinzip
Also Datenvolumen.

Späeter wird er drutlicher und führt das folgende Bild an:\\
Wenn Data-Mart = Wasserflasche\\
Wasser aus Datenqeulle, rest fließt ab \\

\begin{figure}[h]
	\centering 
	\includegraphics[width=0.4\textwidth]{img/p1} 
	\caption[DRT]{DRT \textit{nach} \cite{src6}}	
\end{figure}

Paradigma: Wissen nich wie wertvoll das ist, was da abgeht.
Also: Wasser in See, daraus Data Marts, auch Ad Hoc und DWH\\

\begin{figure}[h]
	\centering 
	\includegraphics[width=0.4\textwidth]{img/p2} 
	\caption[DRT]{DRT \textit{nach} \cite{src6}}	
\end{figure}


Anmerkung zu extra-Pfeilen Data-Mart

Entsprechend folgt die Pentaho-Architektur 2010.

\begin{figure}[h]
	\centering 
	\includegraphics[width=0.4\textwidth]{img/p3} 
	\caption[DRT]{DRT \textit{nach} \cite{src6}}	
\end{figure}

Die drei Schichten werden hier erstmalig gezeigt und sind klar.

Damit wars das an Definition. Niocht sehr genau. Weitere Unterkonzepte und Lösungen entstanden. Es gibt keinen einheitlichen Begriff.\cite{src7}

\section{Wie funktioniert ein Data Lake?}
Begriffe die sich gefestigt haben.

\paragraph{Aufbau und Workflow}
Aufbau
Analog zu Dixon Aufbau in drei Schichten.\cite{src8} \cite{src9}
\begin{itemize}
	\item F
	\item P u S
	\item Viz
\end{itemize}
Dabei wird P und S gelegentlich synonym als Data Lake bezeichnet, was von Dixon abweicht.

Workflow

Zusammengefasst wiefolgt:

\begin{figure}[h]
	\centering 
	\includegraphics[width=0.4\textwidth]{img/p4} 
	\caption[DRT]{DRT \textit{nach} \cite{src9}}	
\end{figure}


Brauchen Daten nach DL\\
Brauchen aufbereiten des Wassers. Hierbei spielt die Rolle des Data Scientist eine Rolle.\cite{src8}\\
Dann zur Verfügung stellen\\
Oben: Visaulisierung. Was dabei visualisiert wird variiert von Sol zu Sol.


\paragraph{Storage}
QUELLEN : 8, ?
\paragraph{Ingestion}
QUELLEN : 8, 10, 11
\paragraph{Process}
QUELLEN : 8, 12
\paragraph{Consumption}
QUELLEN : 8, 12
\paragraph{Monitoring}
QUELLEN : 8, ?
\paragraph{Data Governance}
QUELLEN : 8, 12


\section{Data Swamps: Kritik am Data Lake}
Mögliche Darstellungen die es so gibt:
\begin{itemize}
	\item Sumpf: findest nix und gehst unter \cite{src3}
	\item Finnland: heterogen, nicht zu inregreiren \cite{src13}
	\item Flohmarkt: Findest alles aber wie sucht man?, wem kann man vertrauen? Qualität? \cite{src12}
\end{itemize}

Gartners wesentliche Punkte\\
Aufstieg Data Lake durch scheinbar Löung im Problem. Konzept hat aber Lücken und wenig Substanz. Es kommt zu undergoverned und Meta-Daten-losen Hadoop-Clustern. Dies ist im wesentlichen das was unter dem Begriff Data Swamp verstanden wird.\cite{src3}

Battle: Gartner vs. Dixon und \\
Dixon setzt sich zur Wehr. Insbesondere Anzahl Quellen: Wassergartenarchitektur.\cite{src15} Auch Metadaten. Macht dazu keine Angaben, aber sagt, dass es nicht heißt, dass nicht.\cite{src14} Auf jeden Fall festhalten: ungenaue Definition.\

Neben diesem Problem ansehen, was in der Praxis passiert: Hier ist Sean Martin zu zitieren.\\
Es kommt generell zu dem Trend des Vorsichtiger werdens und die Flut kommen sehen. Paradigmenwechsel.\cite{src1}

\section{Fake-News! Existieren Data Lakes überhaupt?}
Blogeintrag nur nennen und erste Zeile zitieren.\cite{src4}\\

Nach Recherche lassen sich da schon ein paar Firmen finden, die Data Lake Lösungen anbieten.
Unter anderem zu nennen sind : Firma\cite{c1}, Firma\cite{c2}, Firma\cite{c3},...

Nach weiterer Recherche auch success-stories auffindbar. 
Zu nennen sind hierbei die Storys von Firma\cite{ss1}, Firma\cite{ss2}, Firma\cite{ss3}. Auch Zaloni hat Testemonial \cite{ss4}.

Im Bsp von UCI Health ist Lösung gut, weil \cite{src1}\cite{ss2} 

Also irgendwie schon.


Die wesentliche Frage ist allerdings die Definitionsfrage.\\
Selber Schluss im Blogeintrag. Lösung die dem Paradigma grundlegend folgen gibt es. Jetzt im Auge des Betrachters ob man das Kind beim Namen nennt oder nicht.


%------------------------------------------------


%------------------------------------------------



%----------------------------------------------------------------------------------------
%	REFERENCE LIST
%----------------------------------------------------------------------------------------

\bibliographystyle{unsrt}
\bibliography{bib.bib} % The file containing the bibliography

%----------------------------------------------------------------------------------------

\end{document}
